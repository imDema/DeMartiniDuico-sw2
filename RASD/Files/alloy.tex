\subsection{Introduction}
This chapter presents a formal analysis of the application. Since the application domain is time dependent at its core, we decided to model the evolution of the system over time, with focus on the entrances to the Shops. The model focuses on Customers, Tickets and Bookings: the Staff can be seen as an intermediary and moderator between the Customers and the system, therefore, by assuming that the Staff will follow the rules, we can simplify the model for clarity and ease of comprehension.
The scenario being analyzed starts from a state in which Customers already have acquired the tokens and enter the Shops. The queue uses strict constraints, by allowing only the first in queue to enter the Shop. In practice, the constraint can be relaxed to increase throughput if needed. 
In particular, the model checks the following properties:
\begin{itemize}
    \item No department in a Shop exceeds its occupancy limits
    \item Customers cannot enter a Shop without a valid Token
    \item Customers can use a Booking to enter a Shop only at the time specified
    \item Customers cannot cut the waiting line for a Shop
    \item The same Token cannot be used for multiple visits
\end{itemize}
\subsection{Alloy code}
\subsubsection{Model description}
The first section describes the signatures of the objects which are relevant for the formal analysis and their representation invariants.
\lstinputlisting[language=alloy,firstline=1,lastline=68]{alloy/main.als}
\subsubsection{Dynamic properties}
This section describes the rules that guarantee the correct evolution of the system over time. In particular, the \texttt{Trace} fact rules the transition from an instant to the next for Customers and for the Waiting List. 
\lstinputlisting[language=alloy,firstline=72,lastline=127]{alloy/main.als}
\subsubsection{Assertions}
This section lists some of the property that will be guaranteed \emph{if} the invariants described in the previous sections hold. The The result of the evaluation of these assertions will be presented in the next section.
\lstinputlisting[language=alloy,firstline=131,lastline=211]{alloy/main.als}
\subsubsection{Execution results}
This section shows the experimental result of the formal analysis.

\begin{figure}[H]
    \centering
    \includegraphics[width=0.75\textwidth]{Images/Alloy/check_assertions.png}
    \caption{All assertions pass the test on a restricted domain.}
\end{figure}

\pagebreak
\lstinputlisting[language=alloy,firstline=215]{alloy/main.als}

All presented predicates have valid instances in the domain.

The following figures show the evolution of a scenario of the enterAndExit predicate for 5 consecutive Time instants. When a Customer visits a Shop the graph shows a \emph{visiting} relation from the customer to one of the Visit elements, which point to the departments of the shop concerning the visit.
\begin{figure}[H]
    \centering
    \includegraphics[height=0.25\textheight]{Images/Alloy/5Customers_v1_t0.png}
    \caption{Run for Time 0. This is the initial state.}
\end{figure}
\begin{figure}[H]
    \centering
    \includegraphics[height=0.25\textheight]{Images/Alloy/5Customers_v1_t1.png}
    \caption{Run for Time 1. \emph{Customer1} has used his Ticket to enter \emph{Shop0}.}
\end{figure}
\begin{figure}[H]
    \centering
    \includegraphics[height=0.25\textheight]{Images/Alloy/5Customers_v1_t2.png}
    \caption{Run for Time 2. \emph{Customer3} and \emph{4} enter the same shop with a Ticket and a Booking respectively.}
\end{figure}
\begin{figure}[H]
    \centering
    \includegraphics[height=0.25\textheight]{Images/Alloy/5Customers_v1_t3.png}
    \caption{Run for Time 3. \emph{Customer1} correctly exits \emph{Shop0}, while \emph{Customer0} makes use of a Booking for \emph{Shop1}}
\end{figure}
\begin{figure}[H]
    \centering
    \includegraphics[height=0.25\textheight]{Images/Alloy/5Customers_v1_t4.png}
    \caption{Run for Time 4. \emph{Customer4} exits as well.}
\end{figure}

\pagebreak
Figure \ref{fig:waitinglistnode1} and \ref{fig:waitinglistnode2} show a detailed view of the same instance, displaying the working principle of the waiting List. Each \emph{WaitingListNode} represents the position of a Customer in the \emph{Waiting list}, associated with their Ticket. The first in queue for each Shop is marked by the \emph{queue} relation. In the first figure, set at Time 0, we see that \emph{Customer1} is the first in queue, since their Ticket is in the first WaitingListNode for \emph{Shop0} (marked by the \emph{queue} relation), thus he is allowed to enter at the next time frame.
Indeed, in the second figure, set at Time 1, a new \emph{visiting} relation between \emph{Customer1} and \emph{Visit1} is created, to represent the customer visiting \emph{Department1} of \emph{Shop0}. As expected, the \emph{queue} relation of the Shop has now moved forward to the next WaitingListNode, advancing the queue.
\begin{figure}[H]
    \centering
    \includegraphics[width=\textwidth]{Images/Alloy/5Customers_v1_t0_detail_crop.png}
    \caption{The same instance at Time 0, with WaitingListNode elements shown}
    \label{fig:waitinglistnode1}
\end{figure}
\begin{figure}[H]
    \centering
    \includegraphics[width=\textwidth]{Images/Alloy/5Customers_v1_t1_detail_crop.png}
    \caption{The evolution of the instance at Time 1}
    \label{fig:waitinglistnode2}
\end{figure}