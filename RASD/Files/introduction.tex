\subsection{Purpose}

During the COVID-19 pandemic in order to reduce the of transmission of the virus governments have introduced various measures aimed to reduce to a minimum close contacts between people, these measures include social distancing and lockdowns.

During lockdowns people can leave their habitations only for essential needs such as grocery shopping, which makes supermarkets a gathering spot where the virus could spread and consitutes a potential danger for the people visiting.

For this reason the access to essential activities should be limited to lower the density of people inside the activities and allow keeping distances effectively and mitigate the risk.

However this is not of trivial execution, because in order to serve the same amount of people, with reduced maximum capacity, the clients need to access the activity distributed over a longer timespan than regular operation. This means that if too many people want to access the shop in a given moment, some people will be left out and this will result in a line of people waiting to enter the shop.

This is a hazard in and of itself since people outside will stay for a prolonged amount of time in close proximity with each other while waiting for their turn.

The problem of forming lines can be mitigated by handing a number to each client and having people enter in order of arrival. This approach, while better than queueing, still requires all clients to go to the shop and take a ticket. Even though they would not need to stay in a queue, they would need to wait close to the activity nevertheless.

This application aims to provide a digital alternative to the "handing numbers" approach and improve it by allowing people to take their number online, this way people only need to go to the shop when it is their turn to enter and they will not need to hang around the building.

The product will also allow shop managers to effectively monitor the entrances by scanning a QR code associated with the "number" to ensure the safety limits are being followed.

Additionally to the lining up mechanism the application will allow customers to book a visit in the future, which will improve the distribution of customers during the day and the week, since they will be able to choose a less crowded time slot.

CLup will also allow customers to choose the approximate time of their visit and the category of items they wish to buy, which allows making more accurate predictions of the waiting times and, by knowing the areas that they will visit, to better utilize the space in the building, while respecting health and safety measures.

\subsection{Scope}

Using the "The World and The Machine" model by M. Jackson we can give an overall description of the events involved in the project realization, those which cannot be observed by the system: "The World", those strictly related to the system: "The Machine", and those in common between the two.

\begin{figure}[h]
    \centering
    \includegraphics[width=.85\textwidth]{Images/world-machine.pdf}
    \caption{\label{fig:world_machine} World Machine diagram}
\end{figure}

\subsubsection{World phenomena}
\begin{itemize}
    \item Customer enters a Shop
    \item Customer goes to a department
    \item Customer exits a Shop
    \item Customer waits for their turn
    \item Customer travels to the shop
    \item Social distancing
    \item Shop occuapancy
\end{itemize}

\subsubsection{Machine phenomena}
\begin{itemize}
    \item Customer data storage
    \item Ticket queue
    \item Booking data
    \item Department occupancy computation
\end{itemize}

\subsubsection{Shared phenomena}

\paragraph{Controlled by the World}
\begin{itemize}
    \item Registration and login
    \item Manager adds a Shop
    \item Manager configures a Shop
    \item Customer books a visit
    \item Customer gets a ticket
    \item Staff checks a token
    \item Staff emits a substitute ticket
    \item Manager checks Shop occupancy
\end{itemize}

\paragraph{Controlled by the Machine}
\begin{itemize}
    \item System sends a reminder to the Customer
\end{itemize}

\subsubsection{Goals}

\begin{description}
    \item [G1] Customers shall be able to acquire a ticket which grants them access to the Shop
    \item [G2]  Customers shall be able to book a future visit to the Shop
    \item [G3]  Customers shall be able to enter the Shop at the earliest occasion without waiting in line
    \item [G4]  Customers shall be reminded about considering the time needed to travel to the Shop
    \item [G5]  Third Parties shall better exploit the departments of the Shop without breaking social distancing measures
    \item [G6]  Third parties shall be able to monitor the number of visits to the Shop, to ensure compliance with the laws
\end{description}

\subsection{Definitions, Acronyms, Abbreviations}
\subsubsection{Definitions}

\begin{description}
    \item [Third party company] A company that wants to use our services
    \item [Ticket] A digital proof of presence in the waiting list
    \item [Booking] A digital proof of reservation for a visit to the Shop in a specific time slot
    \item [Token] Either a ticket or a booking, it's associated with a code which can also be represented as a QR code
    \item [Staff] Company personnel which is responsible of checking entrances
    \item [Shop] Point of sale of the Third party company
\end{description}
\subsubsection{Acronyms}

\begin{description}
    \item [RASD] Requirement Analysis and Specification Document
    \item [API] Application Programming Interface
    \item [HTTP] Hypertext Transfer Protocol
    \item [HTML] Hypertext Markup Language
    \item [UML] Unified Modeling Language
\end{description}
\subsubsection{Abbreviations}

\begin{description}
    \item [{[Gn]}] n-th goal.
    \item [{[Dn]}] n-th domain assumption.
    \item [{[Rn]}] n-th functional requirement.
\end{description}

\subsection{Revision history}

\begin{description}
    \item[v. 1.0 - 22/12/2020] Initial release
\end{description} 

\subsection{Reference documents}

\begin{description}
    \item [BEEP channel] - Mandatory Project Assignment
    \item [The world \& the machine] - M. Jackson, P. Zave
\end{description}


\subsection{Document structure}
This document is structured in four main chapters, that are as follows:
\begin{enumerate}
    \item[Section 1] The first chapter serves as an introduction and an overview to the project, describing the main reasons for its development, its goals together with a brief informal description of it.

    \item[Section 2] The second chapter serves as a more formal description of the project: it includes class diagrams, state machine diagrams, and it gives details on the shared phenomena and domain models. Class diagrams give a big picture description on how the system should be structured, while state machine diagrams focus on the more relevant entities of the model. Here are also presented all the requirements and domain assumptions the system in project must fulfill and take into considerations, in order to achieve the goals; they are presented each one after the goal it is relevant to.

    \item[Section 3] The third chapter presents the specific requirements: use cases and the design constraints the system must satisfy. The use cases are described using natural language, while the constraints are pictured with sequence/activity diagrams. A mockup is also shown as a general idea of how the end product should be, in terms of design and functionalities offered to the end customer.

    \item[Section 4] The fourth and last chapter is a formal analysis of the model, made through the use of the open source Alloy language and analyzer, including a graphic representation of it obtained from Alloy Tool.
\end{enumerate}
